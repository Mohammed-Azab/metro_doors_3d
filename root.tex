% Milestone 1 Technical Report
% Project: Metro Sliding Doors
% NOTE: This file was auto-generated to structure all required milestone 1 content.
% Fill all TODO placeholders, insert images, and replace assumed values with actual measured/design data.

\documentclass[11pt]{article}
\usepackage[a4paper,margin=1in]{geometry}
\usepackage{graphicx}
\usepackage{booktabs}
\usepackage{longtable}
\usepackage{array}
\usepackage{hyperref}
\usepackage{xcolor}
\usepackage{enumitem}
\usepackage{amsmath,amssymb}
\setlength{\parskip}{0.6em}
\setlength{\parindent}{0pt}
\hypersetup{colorlinks=true,linkcolor=blue,urlcolor=blue}

	\title{Metro Sliding Doors Project}
\author{Student Name: \textit{TODO}\newline Student ID: \textit{TODO}\newline Group: \textit{TODO}}
\date{Submission Date: 08 Oct 2025}

\begin{document}
\maketitle

\section*{Cover Page}
	\textbf{University:} German International University\\
	\textbf{Course:} Mechatronics Lab (MCTR704)\\
	\textbf{Semester:} Berlin Winter 2025\\
	\textbf{Project Title:} Metro Sliding Doors\\
	\textbf{Project No.:} [ x ]\\
	\textbf{Student:} \textit{TODO}\newline
	\textbf{Instructor/Supervisor:} \textit{TODO}\\
\vspace{1em}
	extbf{NOTE:} Insert logo / project representative photo here.\\
\fbox{\parbox{0.9\linewidth}{\centering Image Placeholder (Door System Overall View) -- TODO}}

	ableofcontents
\newpage

% =============================================================
\section{Project Description (Milestone Requirement)}
This section addresses items (a)--(e) of the milestone description. It establishes functional understanding prior to detailed CAD work.

\subsection{(a) Functional Overview}
The \textbf{Metro Sliding Doors} system provides controlled passenger access between train cars/platform and interior space. Two horizontally sliding door panels are actuated pneumatically, ensuring synchronized opening and closing. A third pneumatic cylinder controls a \textbf{safety locking mechanism} that physically prevents door movement when the system is deactivated or unsafe conditions are detected (e.g., train not aligned, emergency stop engaged).\newline
Core functions:
\begin{itemize}[noitemsep]
	\item Provide reliable bi-directional sliding motion with smooth acceleration/deceleration via pneumatic actuation.
	\item Enforce interlocks: doors cannot open unless lock disengages and system is activated from the Master panel.
	\item Indicate door status locally (per door) and centrally (driver cabin panel) using red/green indicators.
	\item Support manual override or emergency closing initiated only from Master panel.
	\item Track door position using reed switches on cylinder end-of-stroke plus optional mid-travel sensors for diagnostics (optional expansion).\end{itemize}

\subsection{(b) Workpiece (Transferred Object) Description}
In this context, the \textbf{``workpiece''} is the \emph{passenger passage aperture} controlled by the two sliding door leaves. For mechanical sizing, each door panel must cover half of the total opening width.\newline
	extbf{Assumed design dimensions (to be validated):}
\begin{itemize}[noitemsep]
	\item Clear opening width (full aperture): \textit{TODO e.g., 1400 mm}
	\item Each sliding leaf nominal width: half aperture + overlap allowance: \textit{TODO}
	\item Door leaf height: \textit{TODO e.g., 2000 mm}
	\item Door panel thickness / construction: aluminum frame with tempered glass insert (color: neutral / light tint) -- \textit{Confirm}\end{itemize}
Color coding (if needed) may use: frame (RAL 9006), glass (clear), safety edges (high-visibility yellow). Replace with actual chosen scheme.\newline
	extbf{NOTE:} Insert detailed dimensional drawing of the door opening and panels here.\\
\fbox{\parbox{0.9\linewidth}{Dimensional Drawing Placeholder (Front Elevation) -- TODO}}

\subsection{(c) Operating Sequence}
High-level sequence from inactive state to cycle completion:
\begin{enumerate}[label=Step \arabic*:,leftmargin=2.2cm]
	\item System powered; Master panel activation button pressed. Safety lock cylinder retracts (unlock). Green indicators turn ON; red indicators OFF.
	\item Operator (driver) or local user presses \emph{Open} command (Master or Slave panel). Both door cylinders retract simultaneously, driving linkages to open panels fully.
	\item Fully open confirmation via reed switches (retracted end). Optional timer or passenger sensor monitors dwell period.
	\item Close command (Master only) issued. Door cylinders extend, sliding leaves shut. Lock cylinder extends only after doors fully closed to engage mechanical locking pawl.
	\item System returns to \emph{secure closed} state; red indicators ON (if deactivated), or remains green if still armed for next cycle without lock engaged (design decision -- clarify).\end{enumerate}
Interlocks:
\begin{itemize}[noitemsep]
	\item Lock must be disengaged before any door motion.
	\item Closing cannot be initiated from Slave panels.
	\item Emergency stop forces air dump, cylinders vent, doors remain (design choice: fail-safe close or fail-freeze). \textit{To define}.\end{itemize}

\subsection{(d) Additional Components for Full Operation}
Beyond base cylinders, full system requires integration of: \newline
	extbf{Actuators and Motion:}
\begin{itemize}[noitemsep]
	\item 2x Pneumatic double-acting cylinders (door motion) with adjustable end cushioning.
	\item 1x Pneumatic cylinder (lock actuator).\end{itemize}
	extbf{Valves and Air Prep:}
\begin{itemize}[noitemsep]
	\item 3x Solenoid-operated 5/2 directional valves (one per cylinder) or manifold with common supply.
	\item FRL unit (Filter-Regulator-Lubricator) + main shutoff valve + pressure gauge.
	\item Quick exhaust valves (optional) for faster closing.\end{itemize}
	extbf{Sensors:}
\begin{itemize}[noitemsep]
	\item Reed switches on cylinder barrels (extended and retracted for each door cylinder and lock cylinder).
	\item Door edge presence / obstruction sensor (photoelectric or light curtain) \textit{(optional safety enhancement)}.
	\item Panel pushbuttons: Open (Master/Slave), Close (Master), Activate, Emergency Stop.
	\item Indicator lights: Green (Ready/Active), Red (Locked/Inactive).\end{itemize}
	extbf{Mechanical Guidance:}
\begin{itemize}[noitemsep]
	\item Linear rails or roller track assemblies for door leaves.
	\item Linkage brackets coupling cylinder rod to carriage.
	\item Locking pawl and strike plate assembly.\end{itemize}
	extbf{Safety / Enclosure:}
\begin{itemize}[noitemsep]
	\item Protective upper compartment housing cylinders and valves.
	\item Panel enclosure (added in Milestone 2) reserved space in frame design.\end{itemize}

\subsection{(e) System Understanding Emphasis}
Prior to CAD work, verify: sizing of cylinders vs required stroke (half door travel), force calculations (friction + inertia), rail selection load rating, lock mechanism sequence timing, and sensor mapping to control logic. \textbf{DO NOT} finalize 3D design until force/stroke assumptions are validated.\newline
	extbf{NOTE:} Insert preliminary engineering calculations (force, stroke, timing) here.\\
\fbox{\parbox{0.9\linewidth}{Engineering Calculation Placeholder (Forces / Stroke / Timing) -- TODO}}

% =============================================================
\subsection*{Mechanical Actuation Mechanism (Provided Details)}
\label{subsec:mechanism}
This project uses a \textbf{dual rack and single pinion} transmission driven by a \textbf{single pneumatic cylinder}. The mechanism is implemented as follows:
\begin{itemize}[noitemsep]
	\item Two parallel racks are mounted horizontally above the door aperture. Each rack is rigidly attached (pinned) to one door leaf.
	\item A pneumatic cylinder is mechanically connected to \emph{Rack A}. When the cylinder extends or retracts, it translates Rack A linearly.
	\item A pinion gear engages both racks. Motion of Rack A drives the pinion, which in turn drives \emph{Rack B} in the opposite linear direction, producing synchronized and symmetric door motion.
	\item The door leaves below run on a bottom rail via \textbf{four wheels per door} (two front, two rear), ensuring guidance, load support, and reduced friction.
	\item End-of-stroke reed switches on the cylinder provide door fully-open and fully-closed confirmations via rack positions (through cylinder travel). Optional mid-travel sensing can be added.
	\item A \textbf{separate locking mechanism} is included and will be described in detail later; its actuation is independent to ensure positive locking when required.
\end{itemize}
Benefits of this arrangement include synchronized leaf motion from a single actuator, compact packaging within the upper compartment, and straightforward position sensing using cylinder-mounted switches.
	extbf{NOTE:} Insert a SolidWorks snapshot of the rack-pinon assembly and the door wheel/rail arrangement.\\
\fbox{\parbox{0.9\linewidth}{Mechanism Figures Placeholder (Racks, Pinion, Cylinder Linkage, Wheels on Rail) -- TODO}}

% =============================================================
\section{SolidWorks 3D Mechanical Design Guidelines (Adapted)}
Design will mirror real hardware implementation. Key project-specific guidelines adapting milestone points:
\begin{itemize}[noitemsep]
	\item Ground-seated frame with vertical uprights supporting upper cylinder compartment.
	\item All pneumatic hardware (valves, FRL, tubing routing) contained within or on rear of frame, not protruding into passenger path.
	\item Frame width sized by aperture + rail assemblies; height sized by door leaf + clearance + cylinder compartment depth.
	\item Reserved mounting plane for future control panel (Milestone 2) on side column.
	\item Clear delineation of stages: (Input = Unlock + Activate), (Operation = Open/Close door motion), (Delivery = Secured locked state ready for next cycle).
	\item Cylinder orientation strictly horizontal; lock cylinder orthogonal or vertical depending design choice (to finalize).
	\item Use linear guide rails for door translation; bearings for any rotating shafts (if conversion mechanism used).
	\item Support layers: base chassis, mid rail support, upper actuator compartment.
	\item Door position tracking via reed switches; optional mid-stroke sensor mount features integrated.\end{itemize}
	extbf{NOTE:} Insert 3D views (isometric, front, side) of assembled model.\\
\fbox{\parbox{0.9\linewidth}{3D Isometric View Placeholder -- TODO}}\\[0.5em]
\fbox{\parbox{0.9\linewidth}{3D Exploded View Placeholder -- TODO}}

% =============================================================
\section{Design For Manufacturing (DFM) Report}
DFM ensures each custom part is feasible with available processes. Provide per-part manufacturing notes and 2D drawings.
\subsection*{Manufacturing Assumptions}
\begin{itemize}[noitemsep]
	\item Frame members: standard rectangular steel/aluminum profiles cut to length, drilled.
	\item Mounting plates: laser-cut sheet metal (specify alloy, thickness), bent where required.
	\item Lock pawl: CNC milled or laser-cut + heat-treated (if wear critical).
	\item Brackets: sheet metal with flange bends, hole patterns for cylinder clevis.
	\item Rails: purchased linear guide assemblies (COTS).\end{itemize}
\subsection*{DFM Part Table}
\begin{longtable}{>{\raggedright}p{0.13\linewidth} >{\raggedright}p{0.22\linewidth} >{\raggedright}p{0.24\linewidth} >{\raggedright}p{0.18\linewidth} >{\raggedright\arraybackslash}p{0.16\linewidth}}
	oprule
Part No. & Part Name & Material / Specs & Manufacturing Process & 2D Drawing Ref. \\
\midrule
\endhead
1 & Base frame side upright & \textit{TODO} & Cut to length, drill & \textit{TODO FIG} \\
2 & Upper cylinder support plate & \textit{TODO} & Laser cut + bend & \textit{TODO FIG} \\
3 & Door carriage bracket & \textit{TODO} & Laser cut + bend & \textit{TODO FIG} \\
4 & Lock pawl & \textit{TODO} & Laser cut + mill finish & \textit{TODO FIG} \\
5 & Valve manifold plate & \textit{TODO} & Laser cut & \textit{TODO FIG} \\
... & ... & ... & ... & ... \\
\bottomrule
\end{longtable}
	extbf{NOTE:} Insert each 2D technical drawing (dimensions, tolerances) following the table.\\
\fbox{\parbox{0.9\linewidth}{2D Drawing Set Placeholder -- TODO}}

% =============================================================
\section{Design For Assembly (DFA) Report}
Focus: minimize assembly time, ensure accessibility, reduce fastener count, enable maintenance.
\subsection*{Assembly Strategy}
\begin{itemize}[noitemsep]
	\item Modular subassemblies: Frame, Door Panels on Carriages, Actuator Compartment (cylinders + valves), Lock Mechanism, Sensor Harness.
	\item Fastener standardization: prioritize M6 socket head and self-locking nuts \textit{(verify)}.
	\item Accessibility: sliding panels removable without disturbing cylinder alignment.
	\item Cable/pneumatic routing channels integrated in upright profiles.
	\item Exploded views to illustrate sequence and tool clearance zones.\end{itemize}
	extbf{NOTE:} Insert exploded subassembly views and annotated assembly sequence list.\\
\fbox{\parbox{0.9\linewidth}{Exploded Subassembly Views Placeholder -- TODO}}

% =============================================================
\section{Mechanical Component List}
Comprehensive inventory per milestone instructions.
\begin{longtable}{>{\raggedright}p{0.05\linewidth} >{\raggedright}p{0.25\linewidth} >{\raggedright}p{0.30\linewidth} >{\raggedright}p{0.10\linewidth} >{\raggedright\arraybackslash}p{0.25\linewidth}}
	oprule
No. & Name & Description / Function & Qty & Notes / Datasheet Ref. \\
\midrule
\endhead
1 & Pneumatic cylinder (door left) & Actuate left sliding door & 1 & Stroke \textit{TODO} \\
2 & Pneumatic cylinder (door right) & Actuate right sliding door & 1 & Stroke \textit{TODO} \\
3 & Lock actuator cylinder & Engage/disengage lock & 1 & Stroke \textit{TODO} \\
4 & 5/2 solenoid valve & Control left door cylinder & 1 & Coil voltage \textit{TODO} \\
5 & 5/2 solenoid valve & Control right door cylinder & 1 & Coil voltage \textit{TODO} \\
6 & 5/2 solenoid valve & Control lock cylinder & 1 & Coil voltage \textit{TODO} \\
7 & FRL unit & Air preparation & 1 & Model \textit{TODO} \\
8 & Pressure regulator gauge & Pressure monitoring & 1 & Range \textit{TODO} \\
9 & Reed switches & End-position sensing & 6 & 2 per cylinder \\
10 & Linear guide rail & Door leaf translation & 2 & Length \textit{TODO} \\
11 & Door carriage assembly & Supports door leaf & 2 & Bearing type \textit{TODO} \\
12 & Door panel (leaf) & Barrier component & 2 & Material \textit{TODO} \\
13 & Lock pawl & Mechanical lock interface & 1 & Hardened? \textit{TODO} \\
14 & Strike plate & Receives lock pawl & 1 & \textit{TODO} \\
15 & Indicator lights (Green) & Status active & 2 & Voltage \textit{TODO} \\
16 & Indicator lights (Red) & Status inactive & 2 & Voltage \textit{TODO} \\
17 & Master panel buttons & Open/Close/Activate & 3 & Type \textit{TODO} \\
18 & Slave panel buttons & Local open & 2 & Type \textit{TODO} \\
19 & Emergency stop & Safety shutdown & 1 & Standard \textit{TODO} \\
20 & Tubing (various diam.) & Pneumatic connections & \textit{m} & Diameter \textit{TODO} \\
21 & Fittings (elbow, T) & Air routing & Set & Count \textit{TODO} \\
22 & Fasteners (M6 bolts) & Structural joints & Set & \textit{TODO} \\
23 & Fasteners (M8 bolts) & High-load joints & Set & \textit{TODO} \\
24 & Cable duct / channel & Wiring/pneumatic mgmt & As req. & Length \textit{TODO} \\
25 & Valve manifold plate & Mount valves & 1 & Material \textit{TODO} \\
... & ... & ... & ... & ... \\
\bottomrule
\end{longtable}
	extbf{NOTE:} Attach PDF datasheets for all purchased components in Appendix (placeholder below).\\
\fbox{\parbox{0.9\linewidth}{Datasheets Appendix Placeholder -- TODO}}

% =============================================================
\section{Pneumatic Position-Step Diagram}
This diagram defines sequence control states of the three cylinders.\newline
State notation: LDC (Left Door Cylinder), RDC (Right Door Cylinder), LC (Lock Cylinder).\newline
	extbf{Legend:} EXT = Extended, RET = Retracted.
\begin{longtable}{>{\raggedright}p{0.10\linewidth} >{\raggedright}p{0.20\linewidth} >{\raggedright}p{0.20\linewidth} >{\raggedright}p{0.20\linewidth} >{\raggedright\arraybackslash}p{0.25\linewidth}}
	oprule
Step & LDC & RDC & LC & Event / Sensor Condition \\
\midrule
\endhead
0 & EXT (Closed) & EXT (Closed) & EXT (Locked) & System inactive (Red ON) \\
1 & EXT & EXT & RET (Unlocked) & Activate pressed (Green ON) \\
2 & RET (Opening) & RET (Opening) & RET & Open command; door travel start (Reed mid optional) \\
3 & RET (Open) & RET (Open) & RET & Both door cylinders retract reed switches ON \\
4 & EXT (Closing) & EXT (Closing) & RET & Close command; lock stays disengaged until doors closed \\
5 & EXT (Closed) & EXT (Closed) & EXT (Lock engage) & Door closed sensors ON, lock extends \\
6 & EXT & EXT & EXT & Cycle complete; ready for next activation \\
\bottomrule
\end{longtable}
	extbf{NOTE:} Insert graphical pneumatic step diagram here.\\
\fbox{\parbox{0.9\linewidth}{Pneumatic Step Diagram Graphic Placeholder -- TODO}}

% =============================================================
\section{Milestone Deliverables Checklist}
\begin{itemize}[noitemsep]
	\item Project description (Section 1) -- \textit{Drafted}
	\item 3D views of mechanical design -- \textit{Pending CAD} (Placeholder inserted)
	\item Mechanical component list -- \textit{Initial list provided; refine quantities}
	\item DFM report + 2D drawings -- \textit{Framework ready; drawings pending}
	\item DFA report with exploded views -- \textit{Framework ready; views pending}
	\item Pneumatic position-step diagram -- \textit{Tabular sequence provided; graphic pending}
	\item SolidWorks 3D design files -- \textit{To include in ZIP upon completion}
\end{itemize}

% =============================================================
\section{Appendix}
\subsection*{A. Datasheets}
	\textbf{NOTE:} Insert PDFs (referenced externally) or summary tables for each purchased component.\\
\fbox{\parbox{0.9\linewidth}{Datasheet Collection Placeholder -- TODO}}
\subsection*{B. Engineering Calculations}
	\textbf{NOTE:} Force sizing for cylinders, friction coefficients, air consumption estimates.\\
\fbox{\parbox{0.9\linewidth}{Calculation Sheets Placeholder -- TODO}}
\subsection*{C. Risk and Safety Notes}
Preliminary safety considerations: pinch points at door edges, emergency stop circuit design, pneumatic pressure limits. Detailed FMEA optional in later milestone.\\
	\textbf{NOTE:} Insert safety assessment here.

\end{document}
