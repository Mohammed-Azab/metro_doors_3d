% Options for packages loaded elsewhere
\PassOptionsToPackage{unicode}{hyperref}
\PassOptionsToPackage{hyphens}{url}
%
\documentclass[
]{article}
\usepackage{amsmath,amssymb}
\usepackage{lmodern}
\usepackage{iftex}
\ifPDFTeX
  \usepackage[T1]{fontenc}
  \usepackage[utf8]{inputenc}
  \usepackage{textcomp} % provide euro and other symbols
\else % if luatex or xetex
  \usepackage{unicode-math}
  \defaultfontfeatures{Scale=MatchLowercase}
  \defaultfontfeatures[\rmfamily]{Ligatures=TeX,Scale=1}
\fi
% Use upquote if available, for straight quotes in verbatim environments
\IfFileExists{upquote.sty}{\usepackage{upquote}}{}
\IfFileExists{microtype.sty}{% use microtype if available
  \usepackage[]{microtype}
  \UseMicrotypeSet[protrusion]{basicmath} % disable protrusion for tt fonts
}{}
\makeatletter
\@ifundefined{KOMAClassName}{% if non-KOMA class
  \IfFileExists{parskip.sty}{%
    \usepackage{parskip}
  }{% else
    \setlength{\parindent}{0pt}
    \setlength{\parskip}{6pt plus 2pt minus 1pt}}
}{% if KOMA class
  \KOMAoptions{parskip=half}}
\makeatother
\usepackage{xcolor}
\IfFileExists{xurl.sty}{\usepackage{xurl}}{} % add URL line breaks if available
\IfFileExists{bookmark.sty}{\usepackage{bookmark}}{\usepackage{hyperref}}
\hypersetup{
  hidelinks,
  pdfcreator={LaTeX via pandoc}}
\urlstyle{same} % disable monospaced font for URLs
\usepackage{longtable,booktabs,array}
\usepackage{calc} % for calculating minipage widths
% Correct order of tables after \paragraph or \subparagraph
\usepackage{etoolbox}
\makeatletter
\patchcmd\longtable{\par}{\if@noskipsec\mbox{}\fi\par}{}{}
\makeatother
% Allow footnotes in longtable head/foot
\IfFileExists{footnotehyper.sty}{\usepackage{footnotehyper}}{\usepackage{footnote}}
\makesavenoteenv{longtable}
\usepackage{graphicx}
\makeatletter
\def\maxwidth{\ifdim\Gin@nat@width>\linewidth\linewidth\else\Gin@nat@width\fi}
\def\maxheight{\ifdim\Gin@nat@height>\textheight\textheight\else\Gin@nat@height\fi}
\makeatother
% Scale images if necessary, so that they will not overflow the page
% margins by default, and it is still possible to overwrite the defaults
% using explicit options in \includegraphics[width, height, ...]{}
\setkeys{Gin}{width=\maxwidth,height=\maxheight,keepaspectratio}
% Set default figure placement to htbp
\makeatletter
\def\fps@figure{htbp}
\makeatother
\setlength{\emergencystretch}{3em} % prevent overfull lines
\providecommand{\tightlist}{%
  \setlength{\itemsep}{0pt}\setlength{\parskip}{0pt}}
\setcounter{secnumdepth}{-\maxdimen} % remove section numbering
\ifLuaTeX
  \usepackage{selnolig}  % disable illegal ligatures
\fi

\author{}
\date{}

\begin{document}

\includegraphics[width=2.01042in,height=0.89415in]{media/image1.png}

\begin{quote}
\textbf{German International University}

\textbf{Mechatronics Lab (MCTR704)}

\textbf{Berlin winter 2025}
\end{quote}

\textbf{TYPE YOUR PROJECT NAME}

\begin{quote}
\textbf{Project No. {[} x {]}}

\textbf{Name:}
\ldots\ldots\ldots\ldots\ldots\ldots\ldots\ldots\ldots\ldots\ldots\ldots\ldots\ldots{}
\textbf{I.D. \#:}\ldots\ldots\ldots. Group: ..\ldots..
\end{quote}

\hypertarget{table-of-contents}{%
\section{\texorpdfstring{Table of Contents
}{Table of Contents }}\label{table-of-contents}}

\begin{longtable}[]{@{}
  >{\raggedright\arraybackslash}p{(\columnwidth - 4\tabcolsep) * \real{0.2359}}
  >{\raggedright\arraybackslash}p{(\columnwidth - 4\tabcolsep) * \real{0.6750}}
  >{\raggedright\arraybackslash}p{(\columnwidth - 4\tabcolsep) * \real{0.0890}}@{}}
\toprule
\begin{minipage}[b]{\linewidth}\raggedright
\begin{quote}
\textbf{Milestone No. / submission date}
\end{quote}
\end{minipage} & \begin{minipage}[b]{\linewidth}\raggedright
\textbf{Content}
\end{minipage} & \begin{minipage}[b]{\linewidth}\raggedright
\begin{quote}
\textbf{Page}
\end{quote}
\end{minipage} \\
\midrule
\endhead
\textbf{{[}1{]}} \textbf{08/10/2025} & Project Description
\textbf{\ldots{}}\ldots.\ldots\ldots\ldots\ldots\ldots\ldots\ldots\ldots\ldots\ldots\ldots\ldots.\ldots\ldots.\ldots\ldots\ldots{}
& \textbf{3} \\
\textbf{{[}1{]}} \textbf{08/10/2025} & Solid works Design: 3D mechanical
design views...\ldots\ldots...\ldots\ldots{} & \\
\textbf{{[}1{]}} \textbf{08/10/2025} & Mechanical Components List & \\
\textbf{{[}1{]}} \textbf{08/10/2025} & DFM report & \\
\textbf{{[}1{]}} \textbf{08/10/2025} & DFA report & \\
\textbf{{[}1{]}} \textbf{08/10/2025} & Mechanical Components in 2D with
Dimensions\ldots\ldots.\ldots\ldots\ldots\ldots\ldots{} & \\
\textbf{{[}1{]}} \textbf{08/10/2025} & Pneumatic position step diagram
\ldots\ldots\ldots\ldots\ldots\ldots\ldots\ldots\ldots\ldots\ldots\ldots.
& \\
\textbf{{[}2{]}} \textbf{22/10/2025} & Electro Pneumatic
Circuit\ldots\ldots\ldots\ldots\ldots\ldots\ldots\ldots\ldots\ldots\ldots\ldots\ldots\ldots\ldots\ldots\ldots.
& \\
\textbf{{[}2{]}} \textbf{22/10/2025} & Electric wiring diagrams (color
coded schematic) \ldots\ldots\ldots\ldots\ldots\ldots\ldots. & \\
\textbf{{[}2{]}} \textbf{22/10/2025} & Design the electric control
circuit using SOLIDWORKS ELECTRICAL 3D & \\
\bottomrule
\end{longtable}

\hypertarget{project-description}{%
\section{\texorpdfstring{Project Description
}{Project Description }}\label{project-description}}

\hypertarget{note}{%
\subsection{\texorpdfstring{Note }{Note }}\label{note}}

\begin{quote}
Before starting your project, you must identify the size (dimensions) of
the workpiece (object) to ensure that your project design is suitable
for the selected object.

Briefly explain your project idea and objectives.

Please note that the project should include a seated frame structure to
support and assemble all components. (Refer to the figure below.)
\end{quote}

\includegraphics[width=5.99653in,height=2.66528in]{media/image2.jpg}

\begin{quote}
Explain the operation of your project based on your mechanical design
and the intended operating process.

Example:

\includegraphics[width=4.75833in,height=3.73333in]{media/image3.jpg}

Solid works Design: 3D Schematic Diagram

\includegraphics[width=0.20667in,height=0.155in]{media/image4.png} Draw
the project in 3D using SolidWorks.

\includegraphics[width=0.20667in,height=0.155in]{media/image4.png}
Create an exploded view of the mechanical 3D model and assign a unique
number to each mechanical part.

\includegraphics[width=0.20667in,height=0.155in]{media/image4.png} Fill
in the table by listing each part number along with its corresponding
part name.
\end{quote}

\begin{longtable}[]{@{}
  >{\raggedright\arraybackslash}p{(\columnwidth - 2\tabcolsep) * \real{0.2460}}
  >{\raggedright\arraybackslash}p{(\columnwidth - 2\tabcolsep) * \real{0.7540}}@{}}
\toprule
\begin{minipage}[b]{\linewidth}\raggedright
\begin{quote}
Part number
\end{quote}
\end{minipage} & \begin{minipage}[b]{\linewidth}\raggedright
\begin{quote}
Name
\end{quote}
\end{minipage} \\
\midrule
\endhead
\begin{minipage}[t]{\linewidth}\raggedright
\begin{quote}
1
\end{quote}
\end{minipage} & Structure level two -- side arm width \\
\begin{minipage}[t]{\linewidth}\raggedright
\begin{quote}
2
\end{quote}
\end{minipage} & Structure level two -- side arm length \\
\begin{minipage}[t]{\linewidth}\raggedright
\begin{quote}
3
\end{quote}
\end{minipage} & Structure level one -- side arm length \\
\begin{minipage}[t]{\linewidth}\raggedright
\begin{quote}
4
\end{quote}
\end{minipage} & Structure support bar \\
\begin{minipage}[t]{\linewidth}\raggedright
\begin{quote}
5
\end{quote}
\end{minipage} & Cylinder A bracket fixing base side \\
\begin{minipage}[t]{\linewidth}\raggedright
\begin{quote}
6
\end{quote}
\end{minipage} & Cylinder A base support \\
\begin{minipage}[t]{\linewidth}\raggedright
\begin{quote}
7
\end{quote}
\end{minipage} & Cylinder A \\
\begin{minipage}[t]{\linewidth}\raggedright
\begin{quote}
8
\end{quote}
\end{minipage} & Cylinder A bracket fixing piston side \\
\bottomrule
\end{longtable}

\hypertarget{mechanical-components-2d-projections-with-dimensions}{%
\section{\texorpdfstring{Mechanical Components 2D Projections with
Dimensions
}{Mechanical Components 2D Projections with Dimensions }}\label{mechanical-components-2d-projections-with-dimensions}}

\begin{quote}
Each part in the project is considered a component, including nuts and
screws.

Each component should be listed in a separate row in the table provided
below.
\end{quote}

\begin{longtable}[]{@{}
  >{\raggedright\arraybackslash}p{(\columnwidth - 0\tabcolsep) * \real{1.0000}}@{}}
\toprule
\begin{minipage}[b]{\linewidth}\raggedright
Fill in the table using the part names and numbers exactly as assigned
in the 3D model.
\end{minipage} \\
\midrule
\endhead
Insert your solid works Drawing in in 2D (with dimensions ) \\
Fill in the table using the part names and numbers exactly as assigned
in the 3D model. \\
Insert your solid works Drawing in in 2D (with dimensions ) \\
\bottomrule
\end{longtable}

\begin{quote}
Extend the table for your entire components list
\end{quote}

\hypertarget{project-components-list-and-data-sheets-description}{%
\section{\texorpdfstring{Project Components List and Data Sheets
Description
}{Project Components List and Data Sheets Description }}\label{project-components-list-and-data-sheets-description}}

\begin{quote}
• Create a table listing all the components required for your project
(both electrical and mechanical).

For each component, include the quantity, a description of its function
and purpose, and a representative photo.

\includegraphics[width=0.20667in,height=0.155in]{media/image5.png}
Attach the PDF datasheets of all selected components to the project
documentation.
\end{quote}

\begin{longtable}[]{@{}
  >{\raggedright\arraybackslash}p{(\columnwidth - 8\tabcolsep) * \real{0.0463}}
  >{\raggedright\arraybackslash}p{(\columnwidth - 8\tabcolsep) * \real{0.1159}}
  >{\raggedright\arraybackslash}p{(\columnwidth - 8\tabcolsep) * \real{0.3684}}
  >{\raggedright\arraybackslash}p{(\columnwidth - 8\tabcolsep) * \real{0.1504}}
  >{\raggedright\arraybackslash}p{(\columnwidth - 8\tabcolsep) * \real{0.3190}}@{}}
\toprule
\begin{minipage}[b]{\linewidth}\raggedright
\end{minipage} & \begin{minipage}[b]{\linewidth}\raggedright
\begin{quote}
\textbf{Name}
\end{quote}
\end{minipage} & \begin{minipage}[b]{\linewidth}\raggedright
\begin{quote}
\textbf{Description}
\end{quote}
\end{minipage} & \begin{minipage}[b]{\linewidth}\raggedright
\textbf{Quantity}
\end{minipage} & \begin{minipage}[b]{\linewidth}\raggedright
\begin{quote}
\textbf{Photo}
\end{quote}
\end{minipage} \\
\midrule
\endhead
\begin{minipage}[t]{\linewidth}\raggedright
\begin{quote}
\textbf{1}
\end{quote}
\end{minipage} & & & & \\
\begin{minipage}[t]{\linewidth}\raggedright
\begin{quote}
\textbf{2}
\end{quote}
\end{minipage} & & & & \\
\begin{minipage}[t]{\linewidth}\raggedright
\begin{quote}
\textbf{3}
\end{quote}
\end{minipage} & & & & \\
\begin{minipage}[t]{\linewidth}\raggedright
\begin{quote}
\textbf{4}
\end{quote}
\end{minipage} & & & & \\
\begin{minipage}[t]{\linewidth}\raggedright
\begin{quote}
\textbf{5}
\end{quote}
\end{minipage} & & & & \\
\begin{minipage}[t]{\linewidth}\raggedright
\begin{quote}
\textbf{.}

\textbf{.}

\textbf{.}

\textbf{.}
\end{quote}
\end{minipage} & & & & \\
\begin{minipage}[t]{\linewidth}\raggedright
\begin{quote}
\textbf{x}
\end{quote}
\end{minipage} & & & & \\
\bottomrule
\end{longtable}

\hypertarget{pneumatic-step-diagram-and-description}{%
\section{\texorpdfstring{Pneumatic Step Diagram and Description
}{Pneumatic Step Diagram and Description }}\label{pneumatic-step-diagram-and-description}}

\begin{itemize}
\item
  Draw the pneumatic step diagram based on your project's operation, as
  shown in the example below.
\item
  Explain your provided pneumatic step diagram; the sequence and the
  project's operation. ➢ Below, an Example for the pneumatic step
  diagram
\end{itemize}

\includegraphics[width=3.83333in,height=2.22847in]{media/image6.jpg}

\hypertarget{electro-pneumatic-circuit}{%
\section{\texorpdfstring{Electro Pneumatic Circuit
}{Electro Pneumatic Circuit }}\label{electro-pneumatic-circuit}}

\begin{itemize}
\item
  Draw the simulated the pneumatic-electrical circuit using the FluidSim
  software application (you can download the software from the
  internet).
\item
  Ensure that the circuit includes all necessary components and operates
  as a fully functional controller for your project.
\end{itemize}

\begin{quote}
\includegraphics[width=4.11in,height=2.09in]{media/image7.png}
\end{quote}

\hypertarget{hardware-model-electrical-mechanical}{%
\section{\texorpdfstring{Hardware Model (Electrical + Mechanical)
}{Hardware Model (Electrical + Mechanical) }}\label{hardware-model-electrical-mechanical}}

\begin{quote}
\textbf{Controller Operating Panel/ Classic Control Implementation}
\end{quote}

\hypertarget{part-a-electrical}{%
\subsection{\texorpdfstring{Part A: Electrical
}{Part A: Electrical }}\label{part-a-electrical}}

\begin{quote}
\textbf{Step 1:} Prepare the list of required materials.

\textbf{Step 2:} Implement the classic control circuit using FluidSim.

\textbf{Step 3:} Design the hardware layout of the classic control
system as a front panel with labels. Use the known specifications of
each component (power supply, solenoids, relays, I/O terminals, etc.) to
determine the appropriate size of the panel for your project.

\textbf{Step 4:} Create the wiring diagram for the components within the
control panel.
\end{quote}

\hypertarget{part-b-mechanical}{%
\subsection{\texorpdfstring{Part B: Mechanical
}{Part B: Mechanical }}\label{part-b-mechanical}}

\begin{quote}
\textbf{Step 1:} Prepare the list of required materials.

\textbf{Step 2:} Assemble all components onto the project's frame
structure, including the electrical control panel.

\textbf{Shown below:} An example of a front panel configuration for
reference.
\end{quote}

\includegraphics[width=5.17639in,height=3.86181in]{media/image8.jpg}

\hypertarget{project-hardware-as-built}{%
\section{\texorpdfstring{Project Hardware As Built
}{Project Hardware As Built }}\label{project-hardware-as-built}}

\begin{quote}
Insert a picture of your project hard ware implementation
\end{quote}

Project Documentation

\end{document}
